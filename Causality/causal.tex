% Created 2013-07-29 Mon 10:26
\documentclass[11pt]{article}
\usepackage[utf8]{inputenc}
\usepackage[T1]{fontenc}
\usepackage{fixltx2e}
\usepackage{graphicx}
\usepackage{longtable}
\usepackage{float}
\usepackage{wrapfig}
\usepackage{soul}
\usepackage{textcomp}
\usepackage{marvosym}
\usepackage{wasysym}
\usepackage{latexsym}
\usepackage{amssymb}
\usepackage{hyperref}
\tolerance=1000
\usepackage{tikz}
\usepackage{attrib}
\usepackage{amsmath}
\let\iint\undefined
\let\iiint\undefined
\usepackage{dsfont}
\usepackage[retainorgcmds]{IEEEtrantools}
\author{Mader, Nick\\ \texttt{Chapin Hall, University of Chicago} \and Misshula, Evan\\ \texttt{Criminal Justice, CUNY Graduate Center}}
\title{Demonstration Of Instrumental Variables And Control Function Methods}
\author{Nick Mader\\ \texttt{Chapin Hall, University of Chicago} \and Evan Misshula\\ \texttt{Criminal Justice, CUNY Graduate Center}}
\date{}
\title{Demonstration Of Instrumental Variables And Control Function Methods}
\hypersetup{
  pdfkeywords={},
  pdfsubject={},
  pdfcreator={Generated by Org mode 8.0-pre in Emacs 24.3.1.}}
\begin{document}

\maketitle
\begin{abstract}
Selection bias affects many of the most promising solutions to social
problems. There are ways to correct for these risks to validity. This
paper discusses Instrumental Variables and Control
Functions. Specifically, This code is a simple demonstration of the
use and implementation of Instrumental Variables (IV) and Control
Function estimation procedures.
\end{abstract}


\section[Quotes]{Quotes}
\label{sec-1}

\begin{quote}
\emph{To parents who despair because their children are unable to master
 the first problems in arithmetic I can dedicate my examples. For, in
 arithmetic, until the seventh grade I was last or nearly last.}
~Jacques Salomon Hadamard (1865-1963)  
\end{quote}

\begin{quote}
\emph{To many, mathematics is a collection of theorems. For me, mathematics
is a collection of examples; a theorem is a statement about a
collection of examples and the purpose of proving theorems is to
classify and explain the examples.}  ~John B. Conway
\end{quote}
\section[Introduction]{Introduction}
\label{sec-2}

To illustrate the benefits of IV and control functions, a well chosen
example is worth far more than a theoretical proof.  So with apologies
to David Hilbert, we will choose a model designed to evoke a common
problem.
\section[Model]{Model}
\label{sec-3}

Let us start with a nested theoretical model:
\begin{equation*}
\begin{array}{rcl}
y & = & [1, x_1, x_2] 
\begin{bmatrix}
\beta_0  \\
\beta_1  \\
\beta_2  \end{bmatrix}
  + \epsilon_w \\
x_1 & = & \mathds{1}_{[-1 + \beta_4 \cdot x_4 + \beta_5 \cdot x_5 + \epsilon_j]}
\end{array}
\end{equation*}  

Here the tuple in the first of these equations $[1, x_1, x_2]$ is our
constant variable, two value categorical variable and continuous
variable.  These variables comprise our input data which have some
unknown linear relationship to an output variable $y$.  The error is
denoted by $e_w$ where $w$ is the index of our principal data records
of interest.  The second equation, is an indicator function which
predicts the value of the first of our inputs.  Here $\epsilon_j$
represents our error in predicting the value of the categorical
variable in the first equation. For computational purposes we can
instantiate our model, using names for our hypothetical variables for
familiarity rather than generalizability, is:

\begin{verbatim}
       wage = 5 + 1.0*JobTraining + 0.3*Act_Rel + e_w
JobTraining = 1[-1 + 0.2*Act_Rel - 0.4*Dist_Mi + e_j > 0]
\end{verbatim}   
In this case we are evaluating a jobs program where we believe that is
having an effect but doubters point out that our results have
selection bias since motivated job seekers are more likely to take the
course in the first place.  In many cases it is critically important
for promising programs to use data to prove their positive effects are
not wholly attributable to selection bias. In some cases randomized
control trials (RCTs) may randomize the bias over both the control and
treatment groups which will allow for unbiased estimation under
ordinary least squares (OLS).  However, in this case taking the course
is a function of distance to the job training center so \texttt{e\_w} and
\texttt{e\_j} are correlated since we can't control for motivation, and since
motivation will determine both wages and the value of
JobTraining. Only by randomizing the number of ``motivated'' job
seekers in the training program can we estimate the actual effect of
the training course.  Note that, for interpretive simplicity,
``Act$_\mathrm{Dev}$'', is interpreted as the relative difference between one's Act
score and the average Act.

Notes on control function approach:

 \begin{IEEEeqnarray*}{rcl}
E[w|X, Z, D]  & = & E[XB + aD + e_w | X, Z, D] \\
              & = & XB + aD + E[e_w | X, Z, D]  \\
              & = & XB + aD + D*(E[e_w | X, Z, D = 1])\\ 
              && +\: (1-D)*(E[e_w | X, Z, D=0]) \\
              & = & XB + aD + D*(E[e_w | e_j > -ZG]) \\
              && +\: (1-D)*(E[e_w | e_j < -ZG])
 \end{IEEEeqnarray*}

See \url{http://en.wikipedia.org/wiki/Inverse_Mills_ratio} for constructions of thesee expectations terms.
% Generated by Org mode 8.0-pre in Emacs 24.3.1.
\end{document}